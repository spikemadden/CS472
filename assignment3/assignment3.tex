\documentclass[letterpaper,10pt,titlepage]{article}

\usepackage{graphicx}
\usepackage{amssymb}
\usepackage{amsmath}
\usepackage{amsthm}

\usepackage{alltt}
\usepackage{float}
\usepackage{color}
\usepackage{url}

\usepackage{balance}
\usepackage[TABBOTCAP, tight]{subfigure}
\usepackage{enumitem}
\usepackage{pstricks, pst-node}

\usepackage{geometry}
\geometry{textheight=8.5in, textwidth=6in}

%random comment

\newcommand{\cred}[1]{{\color{red}#1}}
\newcommand{\cblue}[1]{{\color{blue}#1}}

\usepackage{hyperref}
\usepackage{geometry}

\def\name{Spike Madden}
\title{CS 472 Assignment 3}
\author{Spike Madden}
\date{November 21, 2016}

%% The following metadata will show up in the PDF properties
\hypersetup{
  colorlinks = true,
  urlcolor = black,
  pdfauthor = {\name},
  pdfkeywords = {cs472 `computer architecture''},
  pdftitle = {CS 472 Assignment 3},
  pdfsubject = {CS 472 Assignment 3},
  pdfpagemode = UseNone
}

\begin{document}

\maketitle
\pagebreak

\section*{Part 1}

\subsection*{Paging Calculations}

4KB page \\
32 bit address space \\
10 byte entries \\

(2\^{}32) / (2\^{}12) = 2\^{}20 * 10 = 10.48576 MB \\

\noindent4 KB page \\
64 bit address space \\
10 byte entries \\

(2\^{}64) / (2\^{}12) = 2\^{}52 * 10 = 4.504 * 10\^{}10 MB = 40 PB \\

\noindent8 KB page \\
32 bit address space \\
10 byte entries \\

(2\^{}32) / (2\^{}13) = 2\^{}19 * 10 = 5.24288 MB \\

\noindent8 KB page \\
64 bit address space \\
10 byte entries \\

(2\^{}64) / (2\^{}13) = 2\^{}51 * 10 = 2.252 * 10\^{}10 MB = 20 PB \\

\subsection*{Pipelining}

Pipelining is the concept of having data processing units in series where these units can be executed in parallel. This execution of instructions in a process out of order allows for an increase in efficiency. The RISC pipeline, a five stage execution instruction pipeline, consists of the instruction fetch, instruction decode, execute, memory access and writeback stages. Since all instructions share the same stage,  all 5 steps can be done in a single cycle. It's important to note that pipelining doesn't decrease the time it takes to process a single data element. The increase in efficiency comes from the increase in throughput. With HTTP pipelining, we can send multiple requests without waiting for the response of the first request. We increase the throughput of the system when processing the stream of requests.

\subsection*{IA-32e}

IA-32e is Intel's implementation of x86-64; the 64 bit version of the x86 instruction set. Compared to the previous 32 bit architectures, x86-64 supports up to 2\^{}64 bytes of virtual and physical memory. x86-64 uses 48 bits and legacy mode's physical address extension allows for 52 bit addresses. 64 bit architectures commonly have 3 to 4 page levels to decrease RAM usage. With 32 bits, only 4GB of RAM can be addressed. With physical address extension, Intel added 4 address lines so that 64 GB of RAM could be addressed. Page sizes can also be increased from 4K to 4M with page size extension. A table of page table entries create the page table, whose format is fixed by the hardware. There's also a Translation Lookahead Buffer, or TLB, that is a cache for paging addresses. When TLB is full, older addresses that are least recently used are overwritten.\cite{ia32e}

\bibliographystyle{IEEEtran}
\bibliography{assignment3}
\end{document}
